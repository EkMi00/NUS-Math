\documentclass[12pt]{article}

\usepackage{amsmath}
\usepackage{amssymb}
\usepackage[margin=1in]{geometry}
\usepackage{xcolor}

\title{MA2101 22/23 Sem 1 Finals Solutions}
\author{-}
\DeclareMathOperator{\id}{id}
\DeclareMathOperator{\im}{im}
\DeclareMathOperator{\rk}{rk}
\DeclareMathOperator{\spn}{span}
\DeclareMathOperator{\nullity}{nullity}

\begin{document}
\maketitle

\section*{Question 1}
Note that we have
\begin{align*}
Y
=\begin{pmatrix} y_1\\y_2\\y_3 \end{pmatrix}
&=\begin{pmatrix} 0\\0\\1 \end{pmatrix}(y_3-y_1)
+\begin{pmatrix} 1\\1\\1 \end{pmatrix}(y_1)
+\begin{pmatrix} 0\\1\\0 \end{pmatrix}(y_2-y_1)
\\&=\begin{bmatrix} 0&1&0\\0&1&1\\1&1&0 \end{bmatrix}
\begin{pmatrix} y_3-y_1\\y_1\\y_2-y_1 \end{pmatrix} 
\\&=P
\begin{pmatrix} y_3-y_1\\y_1\\y_2-y_1 \end{pmatrix} 
.
\end{align*}
Likewise 
\[
    Y'
=P\begin{pmatrix} y_3'-y_1'\\y_1'\\y_2'-y_1' \end{pmatrix} 
=P\begin{pmatrix} (y_3-y_1)'\\y_1'\\(y_2-y_1)' \end{pmatrix} 
\]
by linearity of the derivative. Letting $z(x)=y_3(x)-y_1(x)$ and
$w(x)=y_2(x)-y_1(x)$, our differential equation becomes
\[
    P\begin{pmatrix} z'\\y_1'\\w' \end{pmatrix}
    = AP\begin{pmatrix} z\\y_1\\w \end{pmatrix} 
\]
whence multiplying both sides by $P^{-1}$ gives
\[
    \begin{pmatrix} z'\\y_1'\\w' \end{pmatrix}
    = (P^{-1}AP)\begin{pmatrix} z\\y_1\\w \end{pmatrix} 
    =\begin{bmatrix} 1&0&0\\0&2&1\\0&0&2 \end{bmatrix}
    \begin{pmatrix} z\\y_1\\w \end{pmatrix} 
    =\begin{pmatrix} z\\2y_1+w\\2w \end{pmatrix}.
\]
so that $z'=z$, $y_1'=2y_1+w$ and $w'=2w$.
From single-variable calculus, we know the first and third equations have
solutions $z(x)=C_1\exp x$ and $w=C_2\exp{2x}$ (where $C_1,C_2\in\mathbb{R}$ are
arbitrary), so the second equation becomes
\[
    y_1'=2y_1'+w=2y_1+C_2\exp{2x}.
\]
Using the given formula with $p(x)=-2, q(x)=\exp{2x}$,
we get $\mu(x)=\exp{(-2x)}$ and \underline{$y_1=(C_2 x+C_3)\exp{2x}$}
(with $C_3\in\mathbb{R}$ arbitrary.) Finally, we calculate
\[
    \underline{y_2(x)=w(x)+y_1(x)=(C_2x+(C_2+C_3))\exp{2x}},
\]
and
\[
    \underline{y_3(x)=z(x)+y_1(x)=C_1\exp x + (C_2x+C_3)\exp{2x}},
\]
which is exactly what we wanted.
\hfill$\square$

\newpage
\section*{Question 2}
\subsection*{Part (a)}
We claim that $AA^{T}$ is a real symmetric matrix, then by the spectral
theorem $AA^T$ is orthogonally diagonalizable as desired.

Indeed, write $A=(a_{i,j})_{i,j=1}^n$, then $A^T=(a_{j, i})_{i,j=1}^n$, and
\[
    AA^T=(a_{i,j})_{i,j=1}^n(a_{j,i})_{i,j=1}^n
    =\left(\sum_{k=1}^n a_{i,k}a_{j,k}\right)_{i,j=1}^n,
\]
which is clearly symmetric. \hfill$\blacksquare$

\subsection*{Part (b)}
We first calculate
$AA^T=
\begin{bmatrix} 1&1\\-1&-1 \end{bmatrix} 
\begin{bmatrix} 1&-1\\1&-1 \end{bmatrix}
=\begin{bmatrix} 2&-2\\-2&2 \end{bmatrix} $.
Then $AA^T$ clearly has eigenvectors
$\begin{pmatrix} 1\\1 \end{pmatrix} $ (with eigenvalue 0)
and $\begin{pmatrix} 1\\-1 \end{pmatrix}$ (with eigenvalue 4).
Normalising both eigenvectors, we get
$\begin{pmatrix} \frac{1}{\sqrt{2}}\\\frac{1}{\sqrt{2}} \end{pmatrix} $
and $\begin{pmatrix} \frac{1}{\sqrt{2}}\\-\frac{1}{\sqrt{2}} \end{pmatrix} $,
so we can conclude by defining
\[
    \underline{
P=\begin{bmatrix}
    \frac{1}{\sqrt{2}}&\frac{1}{\sqrt{2}}
    \\\frac{1}{\sqrt{2}}&-\frac{1}{\sqrt{2}}
\end{bmatrix},
\qquad
D=\begin{bmatrix} 0&0\\0&4 \end{bmatrix}
}.
\]
\hfill$\square$

\newpage
\section*{Question 3}
\subsection*{Part (a)}
We simply look at all possible cases of elementary divisors of $m_A(x)$.
Certainly we must have $(x-2)^3$ and $(x-8)$ as elementary divisors,
so it suffices to ``make up" elementary divisors with sum of degrees equal to 2.
\begin{itemize}
    \item Case 1:
        $(x-2)$, $(x-2)$, the corresponding JCF is $
        \begin{bmatrix} 
            2& & & & &0\\
             &2& & & & \\
             & &2&1& & \\
             & & &2&1& \\
             & & & &2& \\
            0& & & & &8
        \end{bmatrix}$. 
    \item Case 2:
        $(x-2)^2$, the corresponding JCF is $
        \begin{bmatrix} 
            2&1& & & &0\\
             &2& & & & \\
             & &2&1& & \\
             & & &2&1& \\
             & & & &2& \\
            0& & & & &8
        \end{bmatrix}$. 
    \item Case 3:
        $(x-8)$, $(x-8)$, the corresponding JCF is $
        \begin{bmatrix} 
            2&1& & & &0\\
             &2&1& & & \\
             & &2& & & \\
             & & &8& & \\
             & & & &8& \\
            0& & & & &8
        \end{bmatrix}$. 
    \item Case 4:
        $(x-2)$, $(x-8)$, the corresponding JCF is $
        \begin{bmatrix} 
            2& & & & &0\\
             &2&1& & & \\
             & &2&1& & \\
             & & &2& & \\
             & & & &8& \\
            0& & & & &8
        \end{bmatrix}$. 
\end{itemize}
It is easily verified that we have enumerated all cases, up to
permutation of the Jordan blocks.
\hfill$\square$


\subsection*{Part (b)}
Clearly the minimal polynomial $m_B(x)$ of $B$ is either 
$(x-\lambda_1)^2(x-\lambda_2)$
or $(x-\lambda_1)(x-\lambda_2)^2$.
Just like above, we need to ``make up" a degree of 1 somewhere.

\begin{itemize}
    \item Case 1: $m_B(x)=(x-\lambda_1)^2(x-\lambda_2)$.
        \begin{itemize}
            \item Case 1.1: Elementary divisors are
                $(x-\lambda_1), (x-\lambda_1)^2, (x-\lambda_2)$.
                The corresponding JCF is $
                \begin{bmatrix} 
                    1& & & \\
                     &1&1& \\
                     & &1& \\
                     & & &2
                \end{bmatrix}$. 
            \item Case 1.2: Elementary divisors are
                $(x-\lambda_1)^2, (x-\lambda_2), (x-\lambda_2)$.
                The corresponding JCF is $
                \begin{bmatrix} 
                    1&1& & \\
                     &1& & \\
                     & &2& \\
                     & & &2
                \end{bmatrix}$. 
        \end{itemize}
    \item Case 2: $m_B(x)=(x-\lambda_1)(x-\lambda_2)^2$.
        \begin{itemize}
            \item Case 2.1: Elementary divisors are
                $(x-\lambda_1), (x-\lambda_1), (x-\lambda_2)^2$.
                The corresponding JCF is $
                \begin{bmatrix} 
                    1& & & \\
                     &1& & \\
                     & &2&1\\
                     & & &2
                \end{bmatrix}$. 
            \item Case 2.2: Elementary divisors are
                $(x-\lambda_1), (x-\lambda_2), (x-\lambda_2)^2$.
                The corresponding JCF is $
                \begin{bmatrix} 
                    1& & & \\
                     &2& & \\
                     & &2&1\\
                     & & &2
                \end{bmatrix}$. 
        \end{itemize}
\end{itemize}
Like above, it is easily checked that we have accounted for all
possibilities up to
permutation of the Jordan blocks.
\hfill$\square$

\newpage
\section*{Question 4}
\subsection*{Part (a)}

We note that $T|_{V_1}:V_1\to W$ is a linear map, and that
$\ker T|_{V_1}$ is clearly a subspace of $\ker T$ (because $T$ kills every vector
that $T|_{V_1}$ kills.) Furthermore, $\im T|_{V_1}$ is clearly
a subspace of $W_1$.
Then by the rank-nullity theorem
\begin{align*}
    \dim V_1 &= \dim\im T|_{V_1} + \dim\ker T|_{V_1}
           \\&\color{blue}\leq\color{black}
           \dim W_1 + \dim\ker T|_{V_1}
                                       \\&\color{red}\leq\color{black}
            \dim W_1 + \dim \ker T.
\end{align*}
\hfill$\blacksquare$

\subsection*{Part (b)}
Suppose that $T$ is surjective and observe the inequality above becomes an
equality if we have $\color{red}\ker T|_{V_1}=\ker T$ and
$\color{blue}W_1=\im T|_{V_1}$. Since
$T$ is surjective every vector in $W_1$ is the target of some $\mathbf{v}\in V$
under the mapping $T$,
so the blue condition trivially holds; on the other hand we already
know that $\ker T|_{V_1}\subseteq \ker T$, so it suffices to check the reverse
inclusion.

Let $\mathbf{v}\in\ker T$, then $T(\mathbf{v})=0_W\in W_1$
since $W_1$ is a subspace. By definition
of $V_1$ we must have $\mathbf{v}\in V_1$, so $T|_{V_1}(\mathbf{v})=T(\mathbf{v})
=0_W$ by definition. It
follows that $\mathbf{v}\in\ker T|_{V_1}$, which is exactly what we wanted.
\hfill$\blacksquare$

\newpage
\section*{Question 5}
\subsection*{Part (a)}
Let $\mathbf{v}\in W^\perp$. For any $\mathbf{w}\in W$,
we have
$\langle T(\mathbf{v}),\mathbf{w} \rangle=\langle \mathbf{v},
T^*(\mathbf{w}) \rangle$, but since $W$
is $T^*$-invariant, we have $T^*(\mathbf{w})\in W$. By definition of $W^\perp$,
we thus have $\langle \mathbf{v}, T^*(\mathbf{w}) \rangle = 0$.
Since $\mathbf{w}\in W$ is arbitrary, we have $T(\mathbf{v})\in W^\perp$;
and since $\mathbf{v}\in W^\perp$ is arbitrary, $W^\perp$ is indeed
$T$-invariant.
\hfill$\blacksquare$

\subsection*{Part (b)}
Consider the following counterexample: Let $V=\mathbb{C}^2$ equipped with
the standard inner product, and
$U=\spn\left\{\begin{pmatrix} 1\\0 \end{pmatrix}\right\}$.
It is clear that
$U^\perp=\spn\left\{\begin{pmatrix} 0\\1 \end{pmatrix}\right\}$.
Let $T:V\to V$ be defined by
$T(\mathbf{v})=\begin{bmatrix} 1&1\\1&0 \end{bmatrix} \mathbf{v}$,
then $U$ is clearly $T$-invariant
since $\begin{pmatrix} 1\\0 \end{pmatrix}$ is an eigenvector of $T$; but
we have $\begin{pmatrix} 0\\1 \end{pmatrix} \in U^\perp$ and
$T\begin{pmatrix} 0\\1 \end{pmatrix}=\begin{pmatrix} 1\\1 \end{pmatrix}
\not\in U^\perp$, so $U^\perp$ is not $T$-invariant.
\hfill$\blacksquare$

\newpage
\section*{Question 6}
\subsection*{Part (a)}
We note that the representation of any $\mathbf{w}$ as a linear combination of
$\mathbf{w}_1,\ldots,\mathbf{w}_n$ is unique as 
$\{\mathbf{w}_n,\ldots,\mathbf{w}_n\}$ is a basis; thus our function is
well-defined.

We first verify that it is conjugate symmetric: indeed, 
\begin{align*}
    \left(\sum_{j=1}^ny_j\mathbf{w}_j, \sum_{i=1}^nx_i\mathbf{w}_i\right)
    &=Y^tD\overline{X}
  \\&={\left(Y^tD\overline{X}\right)^t}
      &\text{since $Y^tD\overline X\in\mathbb{C}$}
    \\&=\overline{X}^tD^t(Y^t)^t
    \\&=\overline{X}^tD^tY
    \\&=\overline{\overline{\overline{X}^tD^tY}}
    \\&=\overline{\overline{\overline{X}^t}\overline{D^t}\overline{Y}}
    \\&=\overline{{X}^t\overline{D^t}\overline{Y}}
    \\&=\overline{{X}^tD\overline{Y}}
      \\&=\overline{
    \left(\sum_{i=1}^nx_i\mathbf{w}_i, \sum_{j=1}^ny_j\mathbf{w}_j\right),
}
\end{align*}
where $D=\overline{D^t}$ since positive definite matrices are Hermitian.
To see that our function is linear in the second argument, let us define
the vectors 
$\mathbf{v}=\sum_{i=1}^nx_i\mathbf{w}_i$,
$\mathbf{w}=\sum_{i=1}^ny_i\mathbf{w}_i$, and
$\mathbf{w}+\mathbf{w'}=\sum_{i=1}^ny_i'\mathbf{w}_i$ for complex $x_i,y_i,y_i'$.
Then
$\mathbf{w}+\mathbf{w'}=\sum_{i=1}^n(y_i+y_i')\mathbf{w}_i$, and
by definition
\begin{align*}
    (\mathbf{v},\mathbf{w+w'})
    &=
    \begin{pmatrix} x_1&\ldots&x_n \end{pmatrix} 
    D
    \begin{pmatrix} y_1+y_1'\\\vdots\\y_n+y_n' \end{pmatrix} 
    \\&=
    \begin{pmatrix} x_1&\ldots&x_n \end{pmatrix} 
    D
    \begin{pmatrix} y_1\\\vdots\\y_n \end{pmatrix} 
    +
    \begin{pmatrix} x_1&\ldots&x_n \end{pmatrix} 
    D
    \begin{pmatrix} y_1'\\\vdots\\y_n' \end{pmatrix} 
                       \\&=
    (\mathbf{v},\mathbf{w})+(\mathbf{v},\mathbf{w'}).
\end{align*}

Finally, the fact that our function is positive definite directly follows from
the definition of $D$ being positive definite: let
$\mathbf{v}=\sum_{i=1}^nx_i\mathbf{w}_i$, then
$(\mathbf{v},\mathbf{v})=X^tD\overline{X}\geq0$
with equality if and only if every $x_i$ is 0, which
occurs precisely when $\mathbf{v}=0_W$.
\hfill$\blacksquare$

\newpage
\subsection*{Part (b)}
\subsubsection*{Subpart (i)}
We calculate
\begin{align*}
    \left(\sum_{i=1}^nx_i\mathbf{v}_i, \sum_{j=1}^ny_j\mathbf{v}_j\right)
    &=\sum_{i=1}^nx_i\sum_{j=1}^n\overline{y_j}(\mathbf{v}_i, \mathbf{v}_j)
    \\&=\begin{pmatrix} x_1&\ldots&x_n \end{pmatrix} 
    \begin{pmatrix}
        \sum_{j=1}^n\overline{y_j}(\mathbf{v}_1, \mathbf{v}_j)\\
        \vdots\\
        \sum_{j=1}^n\overline{y_j}(\mathbf{v}_n, \mathbf{v}_j)
    \end{pmatrix}
    \\&=\begin{pmatrix} x_1&\ldots&x_n \end{pmatrix} 
    \begin{bmatrix} (\mathbf{v}_1,\mathbf{v}_1)
        &\ldots
    &(\mathbf{v}_1,\mathbf{v}_n)
        \\ \vdots&\ddots&\vdots\\
    (\mathbf{v}_n,\mathbf{v}_1)&\ldots&(\mathbf{v}_n,\mathbf{v}_n)
    \end{bmatrix} 
    \begin{pmatrix} \overline{y_1}\\\vdots\\\overline{y_n} \end{pmatrix} 
    \\&= X^t A \overline{Y}
.\end{align*}
\hfill$\blacksquare$

\subsubsection*{Subpart (ii)}
 
We have \begin{align*}
    A^*=\overline{A^t}&=\overline{
    \begin{bmatrix} (\mathbf{v}_1,\mathbf{v}_1)
        &\ldots
    &(\mathbf{v}_1,\mathbf{v}_n)
        \\ \vdots&\ddots&\vdots\\
    (\mathbf{v}_n,\mathbf{v}_1)&\ldots&(\mathbf{v}_n,\mathbf{v}_n)
    \end{bmatrix}^t 
    }
                    \\&=
    \begin{bmatrix} \overline{(\mathbf{v}_1,\mathbf{v}_1)}
        &\ldots
        &\overline{(\mathbf{v}_n,\mathbf{v}_1)}
        \\ \vdots&\ddots&\vdots\\
        \overline{(\mathbf{v}_1,\mathbf{v}_n)}
                 &\ldots&\overline{(\mathbf{v}_n,\mathbf{v}_n)}
    \end{bmatrix}
                    \\&=
    \begin{bmatrix} (\mathbf{v}_1,\mathbf{v}_1)
        &\ldots
        &(\mathbf{v}_1,\mathbf{v}_n)
        \\ \vdots&\ddots&\vdots\\
    (\mathbf{v}_n,\mathbf{v}_1)&\ldots&(\mathbf{v}_n,\mathbf{v}_n)
    \end{bmatrix}=A 
\end{align*}
by conjugate symmetry, and likewise $A^t$ is self-adjoint (because transpose
and conjugate operators commute).
\hfill$\blacksquare$

\subsubsection*{Subpart (ii)}
Clearly if $X=0$ then $X^tA\overline{X}=0$. If $X\neq 0$, then
from (b)(i) we know
\[
    X^tA\overline{X}=
    \left(\sum_{i=1}^nx_i\mathbf{v}_i, \sum_{j=1}^nx_j\mathbf{v}_j\right)
    >0
\]
since $\sum_{i=1}^nx_i\mathbf{v}\neq 0$ and inner products are positive definite.
\hfill$\blacksquare$
\end{document}

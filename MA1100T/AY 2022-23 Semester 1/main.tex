\documentclass{article}
\usepackage[utf8]{inputenc}
\usepackage{amsfonts}
\usepackage{amssymb}
\usepackage{amsmath}
\usepackage{amssymb}
\usepackage{amsthm}
\usepackage{mathrsfs}
\usepackage {xcolor}
\usepackage{fullpage}
\usepackage{amsmath,amsfonts,enumerate}

\title{MA1100(T) - Basic Discrete Mathematics (T) Suggested Solutions}
\date{(Semester 1 : AY2022/23)}
\author{Written by : Maximus Pung Jun\newline \\Audited by : Teoh Tze Tzun }
\begin{document}

\maketitle

\section*{Question 1}
\begin{enumerate}[(a)]
\item
\begin{proof}
Define $g:A\rightarrow{\text{Maps}(B,C)}$ by $g(a)=\{(b,c)\in{B}\times{C}:{b\in{B} \text{ and } f(a,b)=c}\}$. First, we want to show that $g$ is a function. See that for all $a\in{A}$ and $b\in{B}$, $f(a,b)$ is well defined, as $f$ is a function. So, it also follows that $g$ is a function. Now, we want to show that $g$ is unique. So, suppose $g_1$ and $g_2$ are functions that map $A$ to $\text{Maps}(B,C)$, such that $g_1(a)(b)=f(a,b)$ and $g_2(a)(b)=f(a,b)$, for all $b\in{B}$. So, one has $g_1(a)=g_2(a)$. Since both $g_1$ and $g_2$ have the same domain, codomain and $f(x)=g(x)$ for all $x\in{A}$, we conclude that $g_1=g_2$ as desired.
\end{proof}
\item \begin{proof}
    Let $f$ be one-to-one. Now suppose $g(a_1)=g(a_2)$. Fix some $b\in{B}$. See that $g(a_1)(b)=g(a_2)(b)$. So, $f(a_1,b)=f(a_2,b)$. Since $f$ is one-to-one, one has $a_1=a_2$ as desired.
\end{proof}
\end{enumerate}

\section*{Question 2}
\begin{proof}
    ($\rightarrow$) Suppose $\text{gcd}(a,b)$ divides $c$. By Bezout's Identity, one can fix integers $n,m\in{\mathbb{Z}}$ such that $\text{gcd}(a,b)=an+mb$. Now fix some integer $k\in\mathbb{Z}$ such that $c=k(an+mb)$. It follows that $-mb=akn-c$. Now let $x=ak$. So, $-mb=ax-c$. Since $-m\in\mathbb{Z}$, it follows that $b$ divides $ax-c$ for some integer $x$.
    \newline
    \\($\leftarrow$) Suppose there exists some $x\in\mathbb{Z}$ such that $b$ divides $ax-c$. Fix some $k\in\mathbb{Z}$ such that $kb=ax-c$. See that $c=ax-kb$. Now observe that $\text{gcd}(a,b)$ divides both $a$ and $b$. Since $x$ and $-k$ are integers, one has $\text{gcd}(a,b)$ divides $ax-kb=c$ as desired.
\end{proof}
\section*{Question 3}
\begin{enumerate}[(a)]
    \item \begin{proof}
        WLOG, suppose $x\leq{y}$. Then $n\cdot{x}\leq n\cdot{y}$. Observe that $\min\{n\cdot{x},n\cdot{y}\}=n\cdot{x}$, and $n\cdot\min\{x,y\}=n\cdot{x}$. So, $\min\{n\cdot{x},n\cdot{y}\}=n\cdot\min\{x,y\}$ as desired.
    \end{proof}
    \item \begin{proof}
        Let $a=\prod\{p^{e_a(p)}:{e_a(p)\neq0}\}$ and $b=\prod\{p^{e_b(p)}:{e_b(p)\neq0}\}$. So, $a^n=\prod\{p^{n\cdot{e_a(p)}}:{e_a(p)\neq0}\}$ and $b^n=\prod\{p^{n\cdot{e_b(p)}}:{e_b(p)\neq0}\}$.
        \begin{equation}
\begin{split}
\text{gcd}(a^n,b^n) & = \prod\{p^{\text{min}\{n\cdot{e_a(p)},n\cdot{e_b(p)\}}}\} \\
 & = \prod\{p^{n\cdot\text{min}\{{e_a(p)},{e_b(p)\}}}\} \\
 & = (\prod\{p^{\text{min}\{{e_a(p)},{e_b(p)\}}}\})^n\\
 & = (\text{gcd}(a,b))^n
\end{split}\notag
\end{equation}
    \end{proof}
\end{enumerate}
\newpage
\section*{Question 4}
Define $f_b:\mathbb{Z}\rightarrow\mathbb{Z}$ by $f_b(a)=a+b$. 
\begin{itemize}
    \item To show that $\pi\circ{f_b}=\pi$, see that $a\sim{f_b(a)}$ for all $a\in\mathbb{Z}$. So, $\pi(a)=\pi\circ{f_b(a)}$. Since the domain of $\pi$ and $\pi\circ{f_b}$ are the same, we conclude that $\pi\circ{f_b}=\pi$ as desired.
    \item It is clear that $f_b\neq\text{id}_\mathbb{Z}$.
    \item To show that $f_b$ is one-to-one, let $f_b(a_1)=f_b(a_2)$. So, $a_1+b=a_2+b$. Thus, one has $a_1=a_2$ as desired.
\end{itemize}
\section*{Question 5}
\begin{enumerate}[(a)]
    \item 
    \begin{proof}
        To show that $G$ is well defined, one can fix some $f\in\bigcup_{n\in\mathbb{N}}\text{Maps}([n],\mathbb{N})$. So, there exists some $n\in\mathbb{N}$ such that $f\in\text{Maps}([n],\mathbb{N})$. So, $f:[n]\rightarrow\mathbb{N}$. Thus, $\text{range}(f)\subseteq\mathbb{N}$ and is finite. So, $G$ is well defined. To show that $G$ is onto, fix some $s\in\mathcal{P}_\text{fin}(\mathbb{N})$. Since $s$ is finite, we can fix some $m\in\mathbb{Z}$ such that $s\approx[m]$. Fix bijection $g:[m]\rightarrow{s}$. We know that $s\subseteq\mathbb{N}$ and that $s$ is finite. Thus, $g\in\text{Maps}([m],\mathbb{N})$. Since $m\in\mathbb{N}$, one has $g\in\bigcup_{n\in\mathbb{N}}\text{Maps}([n],\mathbb{N})$ as desired.
    \end{proof}
    \item 
    \begin{proof}
        Since $\bigcup_{n\in\mathbb{N}}\text{Maps}([n],\mathbb{N})$ is countably infinite, one can fix a bijection $B:\mathbb{N}\rightarrow\bigcup_{n\in\mathbb{N}}\text{Maps}([n],\mathbb{N})$. Observe that $G\circ{B}:\mathbb{N}\rightarrow\mathcal{P}_\text{fin}(\mathbb{N})$ is onto. So, $\mathcal{P}_\text{fin}(\mathbb{N})$ is countably infinite.
    \end{proof}
    \item
    \begin{proof}
        Since $A$ is countably infinite, one can fix a bijection $h:\mathbb{N}\rightarrow{A}$. Now, define $j:\mathcal{P}_\text{fin}(\mathbb{N})\rightarrow\mathcal{P}_\text{fin}(A)$ by $j(X)=\{y\in{A}:h(n)=y,n\in{X}\}$. To show that $j$ is onto, consider some $Y\in\mathcal{P}_\text{fin}(A)$. Now let $X=\{n\in\mathbb{N}:h^{-1}(y)=n,y\in{Y}\}$. Clearly $X\in\mathcal{P}_\text{fin}(\mathbb{N})$. So, $j$ is onto. Since $\mathcal{P}_\text{fin}(\mathbb{N})$ is countably infinite, fix bijection $k:\mathbb{N}\rightarrow\mathcal{P}_\text{fin}(\mathbb{N})$. Observe that $j\circ{k}:\mathbb{N}\rightarrow\mathcal{P}_\text{fin}(A)$ is onto. So, $\mathcal{P}_\text{fin}(A)$ is countably infinite as desired.
    \end{proof}
\end{enumerate}
\section*{Question 6}
\begin{enumerate}[(a)]
    \item \begin{proof}
        The Axiom of Choice states that there is a function $F:P\rightarrow\bigcup{P}$ such that for every $S\in{P}$, $F(S)\in{S}$. We know this as $\phi\notin{P}$. We will show that $F$ is an injection. Let $F(S_1)=F(S_2)$. So, $F(S_1)\in{S_1}$ and $F(S_2)\in{S_2}$. So, we have $F(S_1)\in{S_1}$ and $F(S_1)\in{S_2}$. Since $P$ is a partition, it must be the case that $S_1=S_2$. So, $F$ is an injection from $P$ to $\bigcup{P}$. Thus, $P\preceq\bigcup{P}$ as desired.
    \end{proof}
    \item \begin{proof}
        Let $X=\{\{1\},\{2\},\{1,2\}\}$. Observe that $\bigcup{X}=\{1,2\}$. Now, $|X|=3>2=|\bigcup{X}|$. By the pigeonhole principle, $X\npreceq\bigcup{X}$.
    \end{proof}
\end{enumerate}
\section*{Question 7}
\begin{proof}
    Fix some $x\in\mathbb{R}$. Now define $A=\{y\in\mathbb{Q}:y\leq{x}\}$. We claim that $\text{sup}(A)=x$. By the definition of $A$, see that $y\leq{x}$ for all $y\in{A}$. So, $x$ is clearly an upper bound of $A$.  We will now show that $x$ is the \textbf{least} upper bound of $A$, i.e. the supremum of $A$. For the sake of a contradiction, fix $s\in\mathbb{R}$ such that $\text{sup}(A)=s<x$. Since $s,x\in\mathbb{R}$, there exists a rational $r$ such that $s<r<x$. By the definition of $A$, one has $r\in{A}$. However, we know that $s<r$. So $s$ cannot possibly be an upper bound of $A$. Thus, $\text{sup}(A)=x$ as desired.
\end{proof}
\end{document}

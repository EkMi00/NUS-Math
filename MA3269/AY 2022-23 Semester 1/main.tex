\documentclass{article}
\usepackage[utf8]{inputenc}

\title{MA3269 - Mathematical Finance I Suggested Solutions}
\author{(Semester 1, AY2022/2023)}
\date{Written by: Zhang Jingyi\\Audited by: Chow Yong Lam}

\usepackage{geometry}
\usepackage{amsmath}
\usepackage{amssymb}
\geometry{left=20mm, right=20mm, top=20mm, bottom=20mm}
\begin{document}

\maketitle

\begin{enumerate}
\item[1] 
\begin{itemize}
\item[(a)] 
\begin{itemize}
\item[(i)]
\begin{align*}
U(c)=EU(w_0+X) &= 0.5 \times U(16) + 0.5 \times U(4) \\
1-e^{-c} &= 0.5 \times (1-e^{-16}) + 0.5 \times (1-e^{-4}) \\
e^{-c} &= 0.5 \times (e^{-4}+e^{-16}) \\
c &=4.693141036... \\
&=4.693 \text{(4 s.f.)}
\end{align*}

The certainty equivalent of this game is \$4.693.

$RP=w_0-CE(X;U)=8-4.693141036...=3.306858964...=3.307$ (4 s.f.)

The risk premium of this game is \$3.307.

Since $RP > 0$, i.e., $CE \leq w_0$, the investor should not play the game.\\

\item[(ii)]
The investor should reject the game when $EU(X+w_0)<U(w_0)$.
\begin{align*}
U(w_0) &= U(8)=1-e^{-8} \\
EU(X+w_0) &= p(1-e^{-16})+(1-p)(1-e^{-4}) \\
&< U(w_0) \\
1-e^{-8} &> 1-e^{-4} + (e^{-4} - e^{-16})p \\
e^{-4} - e^{-8} &> p(e^{-4} - e^{-16})\\
p &< \frac{e^{-4}-e^{-8}}{e^{-4}-e^{-16}} \text{\ \ \  since } (e^{-4}-e^{-16})>0 \\
p&<0.9816903928... = 0.9817 \text{(4 s.f.)} \\
\therefore 0&<p<0.9817
\end{align*}

\item[(iii)]
\begin{align*}
U&=1-e^{-w} \\
U'&=e^{-w} \\
U''&=-e^{-w} \\
R_U&=-\frac{U''}{U'}=-(-1)=1 \\
2 \frac{U'}{U}&=\frac{2e^{-w}}{1-e^{-w}} \\
V&=1- \frac{1}{1-e^{-w}} \\
V'&= \frac{e^{-w}}{(1-e^{-w})^2} \\
V''&= - \frac{e^{-w}}{(1-e^{-w})^2} - \frac{2e^{-2w}}{(1-e^{-w})^3} 
\end{align*}
\begin{align*}
R_V&=- \frac{V''}{V'}=\left(\frac{e^{-w}}{(1-e^{-w})^2} + \frac{2e^{-2w}}{(1-e^{-w})^3}\right) \times \frac{(1-e^{-w})^2}{e^{-w}} \\
&=1+ \frac{2e^{-w}}{1-e^{-w}} \\ 
&=R_U+2 \frac{U'}{U} \\
\end{align*}
Since $w>0$, $e^{-w}<1$, $1-e^{-w}>0$, and since $2e^{-w}>0$, $\frac{U'}{U}>0$,
\[R_V=R_U+2 \frac{U'}{U} > R_U\]
Therefore, we can say that investor $B$ is globally more risk averse than investor $A$.\\

\end{itemize}
\item[(b)]
\begin{itemize}
\item[(i)]
Let $c$ be the certainty equivalent of the investment. Since the investment has a convex utility function, 
\begin{align*}
EU(w_0 + X) &\geq U(E(w_0 + X)) = U(w_0+E(X)) > U(w_0) \\
U(c) &> U(w_0) 
\end{align*}
Since the utility function $U$ is strictly increasing, $c>w_0$.

By the definition of risk premium and certainty equivalent, 
\[RP(X;U)=w_0-CE(X;U)<0\]

\item[(ii)]
Let $CE(X;U)$ and $CE(X;V)$ be the certainty equivalents of investors $A$ and $B$ respectively. Since investor $B$ is globally more risk averse than investor $A$, $V(w)=g(U(w))$ where $g$ is an increasing and strictly concave function. Therefore,
\begin{align*}
E(g(U(w_0+X)))&<g(E(U(w_0+X))) \\
E(V(w_0+X))&<g(U(CE(X;U))) \\
V(CE(X;V))&<V(CE(X;U)) 
\end{align*}
Since the utility function $V$ is strictly increasing, $CE(X;V) < CE(X;U)$.

By definitions of certainty equivalent and risk premium, 
\begin{align*}
w_0-CE(X;U)&<w_0-CE(X;V) \\
RP(X;U)&<RP(X;V) 
\end{align*}

\end{itemize}
\end{itemize}
\end{enumerate}

\begin{enumerate}
\item[2]
\begin{itemize}
\item[(a)]
\begin{itemize}
\item[(i)]
Let weights of stocks $A$ and $B$ be $x$ and $(1-x)$ respectively. The risk of the portfolio can be expressed as
\[\sigma_p^2 = 0.04x^2+0.16(1-x)^2+2x(1-x)\rho_{AB}(0.2\times 0.4)\]
To find the weight $x$ when the risk is at its minimum, differentiate $\sigma_p^2$ with respect to $x$
\begin{align*}
\frac{d\sigma_p^2}{dx} &=0.08x-0.32(1-x)+0.16\rho_{AB}(-2x+1) \\
&=(0.4-0.32\rho_{AB})x+(0.16\rho_{AB}-0.32)
\end{align*}

When $\displaystyle\frac{d\sigma_p^2}{dx}=0$, 
\[x=\frac{0.32-0.16\rho_{AB}}{0.4-0.32\rho_{AB}}\]
\newline
Differentiate with respect to $x$ again, $\frac{d^2\sigma_p^2}{dx^2}=0.4-0.32\rho_{AB}$, which is greater than $0$ since $-1\leq\rho_{AB}\leq1$. At $x=\displaystyle\frac{0.32-0.16\rho_{AB}}{0.4-0.32\rho_{AB}}$, the risk $\sigma_p^2$ is at its minimum.
\\
When $x=\displaystyle\frac{0.32-0.16\rho_{AB}}{0.4-0.32\rho_{AB}}$, $(1-x)=\displaystyle\frac{0.08-0.16\rho_{AB}}{0.4-0.32\rho_{AB}}$. \\

The weight vector of the minimum-risk portfolio is 
\[\left(\frac{0.32-0.16\rho_{AB}}{0.4-0.32\rho_{AB}}, \frac{0.08-0.16\rho_{AB}}{0.4-0.32\rho_{AB}}\right)^T\]
\\
\item[(ii)]
Substitute $\rho_{AB}=-0.5$ into the weight vector in 2(a)(i), the weight vector is
\[\left(\frac{0.32+0.08}{0.4+0.16}, \frac{0.08+0.08}{0.4+0.16}\right)^T=\left(\frac{5}{7}, \frac{2}{7}\right)^T\]
The mean of the portfolio is $\frac{5}{7} \times 0.1 + \frac{2}{7} \times 0.15 = \frac{4}{35}$.
\newline
The variance of the portfolio is $0.04 \times (\frac{5}{7})^2 + 0.16 \times (\frac{2}{7})^2+2 \times \frac{5}{7} \times \frac{2}{7} \times 0.2 \times 0.4 \times (-0.5)=\frac{3}{175}$.
\\
\item[(iii)]
Given the variance and the mean rate of return of the minimum-variance portfolio, the equation of the minimum variance frontier can be expressed as (for some value of $a$):
\[\sigma^2=a\left(\mu-\frac{4}{35}\right)^2+\frac{3}{175}\]
To find the value of $a$, substitute the variance and mean corresponding to the weight vector $(0, 1)^T$, where $\mu = 0.15$ and $\sigma^2=0.16$
\begin{align*}
\frac{1}{784}a+\frac{3}{175}&=0.16 \\
a&=112 
\end{align*}
The equation of the minimum variance frontier is
\[\sigma^2=112\left(\mu-\frac{4}{35}\right)^2+\frac{3}{175}\]

\item[(iv)]
The minimum-variance mean is $\frac{4}{35}>0.1$. 
\newline
The smallest variance when the portfolio mean is at least $0.1$ is the variance of the minimum-risk portfolio, which is $\frac{3}{175}$.
\newline
The weight vector is $(\frac{5}{7}, \frac{2}{7})^T$.
\newline
The mean is $\frac{4}{35}$, and the variance is $\frac{3}{175}$.\\
\item[(v)]
The minimum-variance mean is $\frac{4}{35}<0.2$.
\newline
The smallest variance when the portfolio mean is at least $0.2$ is achieved at $\mu = 0.2$.
\newline
The variance is $112(0.2-\frac{4}{35})^2+\frac{3}{175}=\frac{21}{25}$.
\newline
To get the weight of the portfolio, let the weights of stocks $A$ and $B$ be $x$ and $(1-x)$ respectively.
\[0.1x+0.15(1-x)=0.2\]
\[x=-1, \ \ \ \ (1-x)=2\]
The weight vector is $(-1, 2)^T$.
\end{itemize}

\item[(b)]
\begin{itemize}
\item[(i)]
\begin{align*}
a&=\mathbf{1}^T\mathbf{C}^{-1}\mathbf{1}=1+1=2 \\
b&=\mathbf{1}^T\mathbf{C}^{-1}\boldsymbol\mu=-0.1+0.7-0.3=\frac{3}{10} \\
c&=\boldsymbol\mu^{T}\mathbf{C}^{-1}\boldsymbol\mu=-0.02+0.35-0.03=\frac{3}{10} \\
\sigma^2 &= \frac{a\mu^2-2b\mu+c}{ac-b^2} \\
&=\frac{2\mu^2-\frac{3}{5}\mu+\frac{3}{10}}{\frac{51}{100}} \\
&=\frac{200}{51}\mu^2-\frac{20}{17}\mu+\frac{10}{17}
\end{align*}
The equation of the minimum-variance frontier is $\displaystyle\sigma^2=\frac{200}{51}\mu^2-\frac{20}{17}\mu+\frac{10}{17}$.
\\
\item[(ii)]
Completing the squares, $\sigma^2=\frac{200}{51}(\mu-\frac{3}{20})^2+\frac{1}{2}$.
\newline
The mean of the global minimum-variance portfolio is $\frac{3}{20}$ and the variance is $\frac{1}{2}$.
\newline
The weight vector can be found by 
\[\mathbf{w}_{GMV}=\frac{C^{-1}1}{1^TC^{-1}1}=\frac{1}{2}(1,0,1)^T=\left(\frac{1}{2},0,\frac{1}{2}\right)^T\]
\\
\item[(iii)]
Another portfolio that lies on the efficient frontier can be found by
\[\mathbf{w}'=\frac{C^{-1}\mu}{1^TC^{-1}\mu}=\frac{10}{3}(-0.1,0.7,-0.3)^T=\left(-\frac{1}{3},\frac{7}{3},-1\right)^T\]
\\
\item[(iv)]
To check whether the portfolios are efficient, we check whether the weight vector can be expressed as convex combination $\mathbf{w}=\alpha \mathbf{w}_1+(1-\alpha)\mathbf{w}_2$ for some $\alpha \in \mathbb{R}$, where $\mathbf{w}_1=\left(\frac{1}{2},0,\frac{1}{2}\right)^T$ and $\mathbf{w}_2=\left(-\frac{1}{3},\frac{7}{3},-1\right)^T$, and the portfolio mean is more than the GMVP mean, i.e., $\mathbf{w}^T\boldsymbol\mu>\frac{3}{20}$.
\begin{itemize}
\item[(1)]
\[\alpha \left(\frac{1}{2},0,\frac{1}{2}\right)^T+(1-\alpha) \left(-\frac{1}{3},\frac{7}{3},-1\right)^T = \left(\frac{1}{12},\frac{7}{6},-\frac{1}{4}\right)^T\]
There is a solution of $\alpha=\frac{1}{2}$, and $\mathbf{w}^T\boldsymbol\mu=\frac{23}{40}>\frac{3}{20}$, the portfolio is efficient.

\item[(2)]
\[\alpha \left(\frac{1}{2},0,\frac{1}{2}\right)^T+(1-\alpha) \left(-\frac{1}{3},\frac{7}{3},-1\right)^T=\left(\frac{11}{12},-\frac{7}{6},\frac{5}{4}\right)^T\]
There is a solution of $\alpha=\frac{3}{2}$. However, $\mathbf{w}^T\boldsymbol\mu=-\frac{11}{40}<\frac{3}{20}$, the portfolio is not efficient.
\item[(3)]
\[\alpha \left(\frac{1}{2},0,\frac{1}{2}\right)^T+(1-\alpha) \left(-\frac{1}{3},\frac{7}{3},-1\right)^T=\left(-\frac{1}{18},\frac{5}{3},-\frac{11}{18}\right)^T\]
There is no solution for $\alpha$. The portfolio is not efficient.\\

\end{itemize}
\item[(v)]
From the answer obtained in 2(b)(ii), the efficient frontier can be obtained as
\[\sigma^2=\frac{200}{51}\left(\mu-\frac{3}{20}\right)^2+\frac{1}{2}\]
\[\mu=\sqrt{\frac{51}{200}\left(\sigma^2-\frac{1}{2}\right)}+\frac{3}{20}\]

\item[(vi)]
\begin{itemize}
\item[(1)]
From the answer obtained in 2(b)(ii),
\[\sigma^2=\frac{200}{51}\left(\mu-\frac{3}{20}\right)^2+\frac{1}{2}\]
Substitute the value $\mu=\frac{57}{40}$ into the minimum-variance frontier,
\begin{align*}
\sigma^2&=\frac{200}{51}\left(\frac{57}{40}-\frac{3}{20}\right)^2+\frac{1}{2}=\frac{55}{8} \\
\sigma&=\sqrt{\frac{55}{8}}
\end{align*}
Differentiate both sides of the minimum-variance frontier with respect to $\sigma$,
\[2\sigma=\frac{400}{51}\left(\mu-\frac{3}{20}\right)\frac{d\mu}{d\sigma}\]

Substitute the values $\mu=\frac{57}{40}$ and $\sigma=\sqrt{\frac{55}{8}}$,
\begin{align*}
2\sqrt{\frac{55}{8}}&=\frac{400}{51}\left(\frac{57}{40}-\frac{3}{20}\right)\frac{d\mu}{d\sigma} \\
\frac{d\mu}{d\sigma}&=\frac{\sqrt{110}}{20}
\end{align*}
The equation of the Capital Market Line can be obtained by
\begin{align*}
\mu-\frac{57}{40}&=\frac{\sqrt{110}}{20}\left(\sigma-\sqrt{\frac{55}{8}}\right) \\
\mu&=\frac{\sqrt{110}}{20}\sigma+\frac{1}{20}
\end{align*}
Substitute $\sigma=0$, $r_f=\frac{1}{20}$.\
\item[(2)]
As shown in 2(b)(vi)(1), the equation of the Capital Market Line is
\[\mu=\frac{\sqrt{110}}{20}\sigma+\frac{1}{20}\]
\item[(3)]
As shown in 2(b)(vi)(1), the variance of the market portfolio is $\sigma^2=\frac{55}{8}$.
\item[(4)]
\begin{align*}
\mathbf{w}&=\frac{c-b\mu}{ac-b^2}\mathbf{C}^{-1}\mathbf{1}+\frac{a\mu-b}{ac-b^2}\mathbf{C}^{-1}\boldsymbol\mu \\
&=-\frac{1}{4}(1,0,1)^T+5(-0.1,0.7,-0.3)^T \\
&=\left(-\frac{3}{4},\frac{7}{2},-\frac{7}{4}\right)^T
\end{align*}
\end{itemize}

\item[(vii)]
\begin{align*}
\beta_p&=\frac{\sigma_{pm}}{\sigma_m^2} \\
\frac{1}{3}&=\frac{\sigma_{pm}}{\frac{55}{8}} \\
\sigma_{pm}&=\frac{55}{24}
\end{align*}

\item[(viii)]
\[\sigma_{pm}=\sigma_p\sigma_m\rho_{pm}\]
Squaring both sides,
\begin{align*}
\sigma_{pm}^2&=\sigma_p^2\sigma_m^2\rho_{pm}^2 \\
\frac{55}{8}\sigma_p^2\rho_{pm}^2&=\frac{3025}{576}
\end{align*}
Since $\rho_{pm}^2\leq 1$, 
\begin{align*}
\frac{55}{8}\sigma_p^2&\geq\frac{3025}{576} \\
\sigma_p^2&\geq\frac{55}{72}
\end{align*}
\item[(ix)]
$\beta_p=\frac{1}{3}$, beta of the market portfolio $m$ is $\beta_m=1$, beta of the risk-free asset is $\beta_r=0$.
The beta of the globally minimum-variance portfolio is
\[\beta_{GMV}=\frac{\sigma_{GMV}^2}{\sigma_m^2}=\frac{\frac{1}{2}}{\frac{55}{8}}=\frac{4}{55}\]
For the portfolio equally weighted in the aforementioned four components, the beta can be obtained as
\[\beta=\frac{1}{4}\left(\frac{1}{3}+1+0+\frac{4}{55}\right)=\frac{58}{165}\]
\end{itemize}
\end{itemize}
\end{enumerate}

\begin{enumerate}
\item[3]
\begin{itemize}
\item[(a)]
\begin{itemize}
\item[(i)]
\[\mathbf{w}_1=\frac{\mathbf{C}^{-1}\mathbf{1}}{\mathbf{1}^T\mathbf{C}^{-1}\mathbf{1}}, \mathbf{w}_1^T=\frac{\mathbf{1}^T\mathbf{C}^{-1}}{\mathbf{1}^T\mathbf{C}^{-1}\mathbf{1}}\]
\[\mathbf{w}_1^T\mathbf{C}\mathbf{w}_1=\frac{\mathbf{1}^T\mathbf{C}^{-1}\mathbf{C}}{\mathbf{1}^T\mathbf{C}^{-1}\mathbf{1}}\frac{\mathbf{C}^{-1}\mathbf{1}}{\mathbf{1}^T\mathbf{C}^{-1}\mathbf{1}}=\frac{1}{\mathbf{1}^T\mathbf{C}^{-1}\mathbf{1}}=\frac{1}{a}\]
\[\mathbf{w}_2=\frac{\mathbf{C}^{-1}\boldsymbol\mu}{\mathbf{1}^T\mathbf{C}^{-1}\boldsymbol\mu}\]
\[\mathbf{w}_1^T\mathbf{C}\mathbf{w}_2=\frac{\mathbf{1}^T\mathbf{C}^{-1}\mathbf{C}}{\mathbf{1}^T\mathbf{C}^{-1}\mathbf{1}}\frac{\mathbf{C}^{-1}\boldsymbol\mu}{\mathbf{1}^T\mathbf{C}^{-1}\boldsymbol\mu}=\frac{1}{\mathbf{1}^T\mathbf{C}^{-1}\mathbf{1}}=\frac{1}{a}\]
\[\mathbf{w}_2=\frac{\mathbf{C}^{-1}\boldsymbol\mu}{\mathbf{1}^T\mathbf{C}^{-1}\boldsymbol\mu}, \mathbf{w}_2^T=\frac{\boldsymbol\mu^T\mathbf{C}^{-1}}{\mathbf{1}^T\mathbf{C}^{-1}\boldsymbol\mu}\]
\[\mathbf{w}_2^T\mathbf{C}\mathbf{w}_2=\frac{\boldsymbol\mu^T\mathbf{C}^{-1}\mathbf{C}}{\mathbf{1}^T\mathbf{C}^{-1}\boldsymbol\mu}\frac{\mathbf{C}^{-1}\boldsymbol\mu}{\mathbf{1}^T\mathbf{C}^{-1}\boldsymbol\mu}=\frac{\boldsymbol\mu^T\mathbf{C}^{-1}\boldsymbol\mu}{(\mathbf{1}^T\mathbf{C}^{-1}\boldsymbol\mu)^2}=\frac{c}{b^2}\]
\item[(ii)]
\begin{align*}
Cov(r_1,r_2)&=Cov(\alpha \mathbf{w}_1+(1-\alpha)\mathbf{w}_2,\beta \mathbf{w}_1+(1-\beta)\mathbf{w}_2) \\
&=\alpha \beta Cov(\mathbf{w}_1,\mathbf{w}_1)+(1-\alpha)(1-\beta)Cov(\mathbf{w}_2,\mathbf{w}_2)+[\alpha(1-\beta)+\beta(1-\alpha)]Cov(\mathbf{w}_1,\mathbf{w}_2) \\
&=\alpha \beta \sigma_1^2+(1-\alpha)(1-\beta)\sigma_2^2+(\alpha+\beta-2\alpha \beta)Cov(\mathbf{w}_1,\mathbf{w}_2) \\
&=\frac{\alpha \beta}{a}+\frac{(1-\alpha)(1-\beta)c}{b^2}+\frac{\alpha+\beta-2\alpha \beta}{a} \\
&=\frac{\alpha \beta b^2+(1-\alpha)(1-\beta)ac+(\alpha + \beta-2\alpha \beta)b^2}{ab^2} \\
&=\frac{(1-\alpha)(1-\beta)ac-(-\alpha-\beta+\alpha \beta) b^2-b^2+b^2}{ab^2} \\
&=\frac{1}{a}+\frac{(1-\alpha)(1-\beta)(ac-b^2)}{ab^2}
\end{align*}
\end{itemize}
\item[(b)]
\begin{itemize}
\item[(i)]
Since the global minimum-variance portfolio has variance and mean of $\sigma^2_g$ and $\mu_g$ respectively, the minimum-variance frontier has the equation of 
\[\sigma^2=\frac{a}{ac-b^2}(\mu-\mu_g)^2+\sigma_g^2\]
The efficient portfolio $q$ has variance and mean of $\sigma^2_q=\sigma^2_p$ and $\mu_q$, substituting these values into the equation
\[\sigma_p^2=\sigma_g^2+\frac{a}{ac-b^2}(\mu_q-\mu_g)^2\]
\item[(ii)]
From 3(b)(i),
\begin{align*}
\sigma_p^2&=\sigma_g^2+\frac{a}{ac-b^2}(\mu_q-\mu_g)^2 \\
(\mu_q-\mu_g)^2&=\frac{\sigma_p^2-\sigma_g^2}{\frac{a}{ac-b^2}}
\end{align*}
The minimum-variance portfolio $r$ has variance and mean of $\sigma_r^2$ and $\mu_r=\mu_p$, substituting these values into the equation of the minimum-variance frontier
\begin{align*}
\sigma_r^2&=\sigma_g^2+\frac{a}{ac-b^2}(\mu_p-\mu_g)^2 \\
(\mu_p-\mu_g)^2&=\frac{\sigma_r^2-\sigma_g^2}{\frac{a}{ac-b^2}} \\
\Psi_p^2&=\frac{(\sigma_p^2-\sigma_g^2)^2}{(\mu_q-\mu_g)^2}=\frac{\sigma_r^2-\sigma_g^2}{\frac{a}{ac-b^2}} \times \frac{\frac{a}{ac-b^2}}{\sigma_p^2-\sigma_g^2} \\
&=\frac{\sigma_r^2-\sigma_g^2}{\sigma_p^2-\sigma_g^2}
\end{align*}
\end{itemize}
\end{itemize}
\end{enumerate}

\begin{enumerate}
\item[4]
\begin{itemize}
\item[(a)]
\begin{itemize}
\item[(i)]
From $K_3-K_2=K_2-K_1$, $2K_2=K_1+K_3$.
\newline
Suppose for the sake of contradiction that $C_2>\frac{1}{2}(C_1+C_3)$, i.e., $2C_2>C_1+C_3$. 
\newline
To construct an arbitrage strategy, we
\begin{itemize}
    \item Long 1 $K_1$-call
    \item Long 1 $K_3$-call
    \item Short 2 $K_2$-call
\end{itemize}
Let $S_T=$ asset price at time of maturity, $T$, and $r=$ annual interest rate.
\newline
The initial value of the strategy is $C_1+C_3-2C_2<0$ since $2C_2>C_1+C_3$.
\newline
The profit table of the strategy is
\begin{center}
\begin{tabular} { |c|c|c|c|c| }
\hline
  & $S_T<K_1$ & $K_1<S_T<K_2$ & $K_2<S_T<K_3$ & $S_T>K_3$ \\
\hline
long 1 $K_1$-call & $-C_1e^{rT}$ & $S_T-K_1-C_1e^{rT}$ & $S_T-K_1-C_1e^{rT}$ & $S_T-K_1-C_1e^{rT}$ \\
\hline
short 2 $K_2$-call & $2C_2e^{rT}$ & $2C_2e^{rT}$ & $2(K_2-S_T+C_2e^{rT})$ & $2(K_2-S_T+C_2e^{rT})$ \\
\hline
long 1 $K_3$-call & $-C_3e^{rT}$ & $-C_3e^{rT}$ & $-C_3e^{rT}$ & $S_T-K_3-C_3e^{rT}$ \\
\hline
\end{tabular}
\end{center}
When $S_T<K_1$, total profit is $-C_1e^{rT}+2C_2e^{rT}-C_3e^{rT}=e^{rT}(2C_2-C_1-C_3)>0$ since $2C_2>C_1+C_3$.\\
When $K_1<S_T<K_2$, total profit is $S_T-K_1-C_1e^{rT}+2C_2e^{rT}-C_3e^{rT}=e^{rT}(2C_2-C_1-C_3)+S_T-K_1>0$ since $2C_2>C_1+C_3$ and $S_T>K_1$.\\
When $K_2<S_T<K_3$, total profit is $S_T-K_1-C_1e^{rT}+2K_2-2S_T+2C_2e^{rT}-C_3e^{rT}=e^{rT}(2C_2-C_1-C_3)+2K_2-K_1-S_T=e^{rT}(2C_2-C_1-C_3)+K_1+K_3-K_1-S_T=e^{rT}(2C_2-C_1-C_3)+K_3-S_T>0$ since $2C_2>C_1+C_3$ and $K_3>S_T$.
\\
When $S_T>K_3$, total profit is $S_T-K_1-C_1e^{rT}+2K_2-2S_T+2C_2e^{rT}+S_T-K_3-C_3e^{rT}=e^{rT}(2C_2-C_1-C_3)+2K_2-K_1-K_3=e^{rT}(2C_2-C_1-C_3)>0$ since $2C_2>C_1+C_3$ and $2K_2-K_1-K_3=0$.
\\
This is an arbitrage opportunity. Therefore, it can be proved that $C_2\leq\frac{1}{2}(C_1+C_3)$.
\item[(ii)]
Let $S_T=$ asset price at time of maturity, $T$, and $r=$ annual interest rate.
\newline
By put-call parity,
\begin{align*}
C(K)-P(K)&=S_T-Ke^{-rT} \\
P(K)&=C(K)-S_T+Ke^{-rT} \\
2P_2&=2C_2-2S_T+2K_2e^{-rT} \\
&\leq(C_1+C_3)-2S_T+2K_2e^{-rT} \text{ \ \ since } C_2\leq\frac{1}{2}(C_1+C_3) \text{ as proven in 4(a)(i)} \\
&=C_1+C_3-2S_T+e^{-rT}(K_1+K_3) \text{\ \ \ since } 2K_2=K_1+K_3 \\
&=(C_1-S_T+K_1e^{-rT}+(C_3-S_T+K_3e^{-rT}) \\ 
&=P_1+P_3 \text{ \ \ by put-call parity} \\
2P_2 &\leq P_1+P_3 \\
P_2 &\leq \frac{1}{2}(P_1+P_3)
\end{align*}

\end{itemize}
\item[(b)]
\begin{itemize}
\item[(i)]
From the question, $S_0=19$, $K=20$, $C=1$, $P=1.5$, $T=\frac{1}{4}$, $r=0.04$.
\begin{align*}
    C-P&=1-1.5=-0.5\\
    S_0-Ke^{-rT}&=10-20e^{-0.04\times \frac{1}{4}}=-0.800996675\\
    C-P&>S_0-Ke^{-rT}\\
    C+Ke^{-rT}&>S_0+P
\end{align*}

To construct an arbitrage strategy, we
\begin{itemize}
    \item Long 1 share
    \item Long 1 $K$-put
    \item Short $Ke^{-rT}$ worth of risk-free asset
    \item Short 1 $K$-call
\end{itemize}
Initial value is $S_0+P-(C+Ke^{-rT})<0$.
\newline
Profit matrix is
\begin{center}
\begin{tabular} { |c|c|c| }
\hline
 & $S_T<20$ & $S_T>20$ \\
 \hline
 Long 1 share & $S_T-S_0e^{rT}$ & $S_T-S_0e^{rT}$ \\
 \hline
 Long 1 $K$-put & $20-S_T-Pe^{rT}$ & $-Pe^{rT}$ \\
 \hline
 Short $Ke^{-rT}$ worth of risk-free asset & 0 & 0 \\
 \hline
 Short 1 $K$-call & $Ce^{rT}$ & $Ce^{rT}-S_T+20$ \\
 \hline
 \end{tabular}
\end{center}
When $S_T<20$, total profit is $S_T-S_0e^{rT}+20-S_T-Pe^{rT}+Ce^{rT}=e^{0.04\times \frac{1}{4}}(1-1.5)+20-19e^{0.04\times \frac{1}{4}}=0.3040217419>0$.
\newline
When $S_T>20$, total profit is $S_T-S_0e^{rT}-Pe^{rT}+Ce^{rT}-S_T+20=e^{0.04\times \frac{1}{4}}(1-1.5)+20-19e^{0.04\times \frac{1}{4}}=0.3040217419>0$.
\newline
This is an arbitrage strategy.
\\
\item[(ii)]
From the question, $S_0=80$, $K=75$, $r=0.1$, $T=\frac{1}{2}$, $C_E=8$.
\begin{align*}
    S_0-Ke^{-rT}&=80-75e^{-0.1\times \frac{1}{2}}=8.657793162\\
    C_E&=8\\
    S_0-Ke^{-rT}&>C_E\\
    S_0&>Ke^{-rT}+C_E
\end{align*}
To construct an arbitrage strategy, we
\begin{itemize}
    \item Short 1 share
    \item Long $Ke^{-rT}$ worth of risk-free asset
    \item Long 1 $K$-call
\end{itemize}
Initial value is $Ke^{-rT}+C_E-S_0<0$.
\\
Profit matrix is
\begin{center}
\begin{tabular} { |c|c|c| }
\hline
 & $S_T<75$ & $S_T>75$ \\
 \hline
 Short 1 share & $-S_T+S_0e^{rT}$ & $-S_T+S_0e^{rT}$ \\
 \hline
 Long $Ke^{-rT}$ worth of risk-free asset & 0 & 0 \\
 \hline
 Long 1 $K$-call & $-C_Ee^{rT}$ & $-C_Ee^{rT}+S_T-75$ \\
 \hline
 \end{tabular}
\end{center}
When $S_T<75$, total profit is $-S_T+S_0e^{rT}-C_Ee^{rT}=80e^{0.1\times \frac{1}{2}}-8e^{0.1\times \frac{1}{2}}-S_T=75.69151894-S_T>0$ since $S_T<75$.
\newline
When $S_T>75$, total profit is $80e^{0.1\times \frac{1}{2}}-75-8e^{0.1\times \frac{1}{2}}=0.69151894>0$.
\newline
This is an arbitrage strategy.\\
\item[(iii)]
From the question, $S_0=58$, $K=65$, $r=0.05$, $T=\frac{1}{6}$, $P_E=6$.
\[P_E=6\]
\[Ke^{-rT}-S_0=65e^{-0.05\times \frac{1}{6}}-58=6.460584022\]
\[P_E<Ke^{-rT}-S_0\]
\[P_E+S_0<Ke^{-rT}\]
To construct an arbitrage strategy, we
\begin{itemize}
    \item Short $Ke^{-rT}$ worth of risk-free asset
    \item Long 1 $K$-put
    \item Long 1 share
\end{itemize}
Initial value is $P_E+S_0-Ke^{-rT}<0$.
\newline
Profit matrix is
\begin{center}
\begin{tabular} { |c|c|c| }
\hline
 & $S_T<65$ & $S_T>65$ \\
 \hline
 Short $Ke^{-rT}$ worth of risk-free asset & 0 & 0 \\
 \hline
 Long 1 $K$-put & $65-S_T-P_Ee^{rT}$ & $-P_Ee^{rT}$ \\
 \hline
 Long 1 share & $S_T-S_0e^{rT}$ & $S_T-S_0e^{rT}$ \\
 \hline
 \end{tabular}
\end{center}
When $S_T<65$, total profit is $65-6e^{0.05 \times \frac{1}{6}}-58e^{0.05 \times \frac{1}{6}}=0.4644382587$.
\newline
When $S_T>65$, total profit is $S_T-6e^{0.05 \times \frac{1}{6}}-58e^{0.05 \times \frac{1}{6}}=S_T-64.53556174>0$ since $S_T>65$.
\newline
This is an arbitrage strategy.
\end{itemize}
\end{itemize}
\end{enumerate}

\end{document}
